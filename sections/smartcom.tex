\subsection{Communication Middleware}
\subsubsection{Functionality and Features}
SmartCom provides communication functionality between a Hybrid Diversity-Aware Collective Adaptive System platform (HDA-CAS) on one side, and ICUs (Individual Computing Units, i.e., human-based services and software-based services) on the other side. SmartCom provides low-level communication and control primitives that effectively virtualize the peers to HDA-CAS platform. It offers the asynchronous (message-based) communication functionality for interacting with dynamically-evolving collectives of peers both through native HDA-CAS applications, as well as through various third-party tools, such as Dropbox, Android devices, Twitter or email clients.
\subsubsection{Implementation}
The CM is implemented in java.
\subsubsection{Interfaces, Endpoints and Resources Exposed}
REST API

GET	`/?type=<type>&subtype=<subtype>`	Get the message information for a message with a specific type and subtype
POST	`/`	Add new message information for a message with a specific type and subtype
GET /?type=<type>&subtype=<subtype>

Returns the message information for a message with a specific type and subtype.

Parameter:
type required - defines the type of the message subtype required - defines the subtype of the message

Respond:
200 message information instance is returned in the body
404 there is no message information for this type and subtype combination

POST /

Create a new message information for a given type and subtype (encoded in the body). If there is already such a message information, it will be overridden with this information.

Parameter:
body required - instance of a message information JSON object

Respond:
200 resource has been created and the created message information is returned in the body

/**
     * Send a message to a collective or a single peer. The method assigns an ID to the message and
     * handles the sending asynchronously, i.e., it returns immediately and does not wait for the
     * sending to succeed or fail. Errors and exceptions thereafter will be sent to the Notification
     * Callback. Optionally, receipt acknowledgements are communicated back through the Notification
     * Callback API.
     *
     * The receiver of the message is defined within the message, it can be a peer or a collective.
     *
     * @param message Specifies the message that should be handled by the middleware.  The receiver of the message is
     *                defined by the message.
     * @return Returns the internal ID of the middleware to track the message within the system.
     * @throws CommunicationException a generic exception that will be thrown if something went wrong
     *                                in the initial handling of the message.
     */
    public Identifier send(Message message) throws CommunicationException;

    /**
     * Add a special route to the routing rules (e.g., route input from peer A
     * always to peer B). Returns the ID of the routing rule (can be used to delete it).
     * The middleware will check if the rule is valID and throw an exception otherwise.
     *
     * @param rule Specifies the routing rule that should be added to the routing rules of the middleware.
     * @return Returns the middleware internal ID of the rule
     * @throws InvalidRuleException if the routing rule is not valid.
     */
    public Identifier addRouting(RoutingRule rule) throws InvalidRuleException;

    /**
     * Remove a previously defined routing rule identified by an Id. As soon as the method returns
     * the routing rule will not be applied any more. If there is no such rule with the given Id,
     * null will be returned.
     *
     * @param routeId The ID of the routing rule that should be removed.
     * @return The removed routing rule or null if there is no such rule in the system.
     */
    public RoutingRule removeRouting(Identifier routeId);

    /**
     * Creates a input adapter that will wait for push notifications or will pull for updates in a
     * certain time interval. Returns the ID of the adapter.
     *
     * @param adapter Specifies the input push adapter.
     * @return Returns the middleware internal ID of the adapter.
     */
    public Identifier addPushAdapter(InputPushAdapter adapter);

    /**
     * Creates a input adapter that will pull for updates in a certain time interval.
     * Returns the ID of the adapter. The pull requests will be issued in the specified
     * interval until the adapter is explicitly removed from the system.
     *
     * @param adapter  Specifies the input pull adapter
     * @param interval Interval in milliseconds that specifies when to issue pull requests. Can’t be zero or negative.
     * @return Returns the middleware internal ID of the adapter.
     */
    public Identifier addPullAdapter(InputPullAdapter adapter, long interval);

    /**
     * Creates a input adapter that will pull for updates in a certain time interval.
     * Returns the ID of the adapter. The pull requests will be issued in the specified
     * interval. If deleteIfSuccessful is set to true, the adapter will be removed in case of
     * a successful execution, it will continue in case of a unsuccessful execution.
     *
     * @param adapter  Specifies the input pull adapter
     * @param interval Interval in milliseconds that specifies when to issue pull requests. Can’t be zero or negative.
     * @param deleteIfSuccessful delete this adapter after a successful execution
     * @return Returns the middleware internal ID of the adapter.
     */
    public Identifier addPullAdapter(InputPullAdapter adapter, long interval, boolean deleteIfSuccessful);

    /**
     * Removes a input adapter from the execution. As soon as this method returns, the
     * adapter with the given ID will not be executed any more. It returns the requested
     * input adapter or null if there is no adapter with such an ID in the system.
     *
     * @param adapterId The ID of the adapter that should be removed.
     * @return Returns the input adapter that has been removed or nothing if there is no such adapter.
     */
    public InputAdapter removeInputAdapter(Identifier adapterId);

    /**
     * Registers a new type of output adapter that can be used by the middleware to get in contact with a peer.
     * The output adapters will be instantiated by the middleware on demand. Note that these adapters are required
     * to have an @Adapter annotation otherwise an exception will be thrown.
     * In case of a stateless adapter, it is possible that the adapter will be instantiated immediately. If
     * any error occurs during the instantiation, an exception will be thrown
     *
     * @param adapter The output adapter that can be used to contact peers.
     * @return Returns the middleware internal ID of the created adapter.
     * @throws CommunicationException if the adapter could not be handled, the specific reason is embedded in
     *                                the exception.
     * @see at.ac.tuwien.dsg.smartcom.adapter.annotations.Adapter
     */
    public Identifier registerOutputAdapter(Class<? extends OutputAdapter> adapter) throws CommunicationException;

    /**
     * Removes a type of output adapters. Adapters that are currently in use will be removed
     * as soon as possible (i.e., current communication won’t be aborted and waiting messages
     * in the adapter queue will be transmitted).
     *
     * @param adapterId Specifies the adapter that should be removed.
     */
    public void removeOutputAdapter(Identifier adapterId);

    /**
     * Register a notification callback that will be called if there are new input messages
     * available.
     *
     * @param callback callback for notification
     * @return returns the identifier of the callback (can be used to remove it)
     */
    public Identifier registerNotificationCallback(NotificationCallback callback);

    /**
     * Unregister a previously registered notification callback.
     *
     * @param callback callback for notification
     */
    public boolean unregisterNotificationCallback(Identifier callback);
\subsubsection{Repository}
\url{https://gitlab.com/smartsociety/SmartCom/}
