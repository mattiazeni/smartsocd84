In this section we will list the components integrated in the current
version of the SmartSociety platform and the key APIs thereof
used for supporting the two scenarios described in the previous
section.

\subsection{Peer Manager}
%The Peer Manager (PM) is developed within WP4 and its description can
%be found in Deliverable D4.2~\cite{D4.2} . 
The Peer Manager APIs specifications can be found at
\url{http://demos.disi.unitn.it:8080/smartsociety-score-api/} and are further
described in Deliverable D4.2~\cite{D4.2}. The PM is used for:
\begin{itemize}
\item {\bf Retrieving peers satisfying some requirements}: This
  request is made by the Orchestration Manager in order to identify
  potential peers composing a suitable collective for carrying out a
  given task. Endpoint used:\\
	\url{/instances/search/} (\textsc{GET})
\item {\bf Creating collective}: This request is made by the Application/Application Runtime at the end of the negotiation. Endpoint used:\\
	\url{/instances/} (\textsc{POST})
\item {\bf Retrieving peers in a collective}: This request is made by
  the Smartcom CM, so that it will be able to retrieve peer information later. Endpoint used:\\
	\url{/instances/search} (\textsc{GET})
\item {\bf Retrieving peers info}: This request is made by Smartcom, that requires the information about the communication channel to be used. Endpoint used:\\
	\url{/instances/search} (\textsc{GET})	
\end{itemize}


\subsection{Communication Middleware}
%The SmartCom Communication Middleware is developed within WP7 and its
%description can be found in Deliverable D7.1~\cite{D7.1}. 
The SmartCom Communication Middleware APIs specifications can be found at 
\url{https://github.com/tuwiendsg/SmartCom} and are further described in Deliverable D7.1~\cite{D7.1}. The Communication Middleware is mainly used for:
\begin{itemize}
\item {\bf Application to peer communication}: The Smart Society application sends messages to the peers belonging to a collective through Smartcom, that will deliver it using the preferred communication channel declared in the peer profile.
\item {\bf Peer to application reply communication}: Peers can reply to messages from the application. The platform provides an endpoint to which the peers can send a reply to a given message. The message is identified by the conversation id, this attribute is also used to know which application.
originally created the message, so that the platform can correctly dispatch it. 
\end{itemize}

\subsection{Orchestration Manager}
%The Orchestration Manager (OM) is developed within WP6 and its
%description can be found in  Deliverable D6.2~\cite{D6.2}. 
The Orchestration Manager APIs specifications can be found at
\url{https://bitbucket.org/rovatsos/smartsociety-internal/wiki/PeerManager/APIBGU}
and are further described in Deliverable D6.2~\cite{D6.2}. \\
In general, the various operations that the OM supports can be divided
into two big groups, depending on whether they require synchronisation
on the back-end or not:
\begin{itemize}
\item Operations that are read-only, and thus no synchronisation is
  needed on the documents on the backend;\
\item Operations that require write-operations, and thus
  synchronisation is critical and should happen on the backend
  (potentially by using semaphores and other approaches developed
  within WP6).
\end{itemize}
The key OM operations are:
\begin{itemize}
\item {\bf Posting a new task request}: In order to post a task request the following endpoint is used:\\
	\url{/applications/:app/taskRequests} (\textsc{POST})
\item {\bf Querying task requests:} The Smart Society application polls the orchestration manager to know the composition/negotiation state of the request. Endpoint used:\\
  \url{/applications/:app/taskRequests/:request_id} (\textsc{GET})
\item {\bf Querying tasks:} The Smart Society application retrieve the negotiable tasks to know which peer to contact for the negotiation. Endpoint used:\\
  \url{/applications/:app/tasks/:task_id} (\textsc{GET})
\item {\bf Updating tasks:} The Smart Society application updates the task according to whether the peer accepted or rejected it. Endpoint used:\\
  \url{/applications/:app/negotiations} (\textsc{PUT})
\end{itemize}
