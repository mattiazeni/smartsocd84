\subsection{Orchestration Manager}
\subsubsection{Functionality and Features}
The OM provides collective composition and negotiation functionality. They can be invoked subsequently or jointly (in the latter case we talk about continuous orchestration, see~\cite{D6.2} for a detailed description. The OM takes as input task requests and outputs an agreed execution plan. 
\subsubsection{Implementation}
The OM implementation shipped with the toolbox is written in Javascript and based upon the node.js framework. It uses express for REST APIs, jade as node template and includes a mongoDB instance for persistency. Two versions are provided, supporting AskSmartSociety! and SmartShare applications. They can be easily reused to develop application-specific OMs.
\subsubsection{Interfaces, Endpoints and Resources Exposed}
\todo{taken from D2.3 - appendix}
For convenience we split the API into different sections.
B.1.1 Task Requests
We start with task requests. The most basic operations are listed in the following table:
 
verb
URI
POST
/applications/:app/taskRequests
GET
/applications/:app/taskRequests/?user=:user
GET
/applications/:app/taskRequests/:taskRequestID
HEAD
/applications/:app/taskRequests/:taskRequestID
GET
/applications/:app/taskRequests/:taskRequestID/v/:version
DELETE
/applications/:app/taskRequests/:taskRequestID
Create Task Request: POST /applications/:app/taskRequests This is the main URI where new task requests are posted. The JSON object describing the task request is expected in the body of the request. On success a platform call to the composition manager will be made.
Access Control. Any peer or user.
Success. Returns error code 201 together with
• a JSON document of the form {
28 of 57
http://www.smart-society-project.eu
Deliverable D2.3 ⃝c SmartSociety Consortium 2013-2017 data: aURI
}
where aURI is the URI where the client can retrieve (assuming authentication and access control policies have no issues) the latest version of the task request that has been posted, and optionally
• an ETag for the JSON object of the response.


Get Task Requests of User:
GET /applications/:app/taskRequests/?user=:user apart from authentication purposes (in the header).
Access Control. The peer user or an admin. Success. Returns error code 200 together with
• a JSON document of the form {
No parameters are expected
      data: [[userTaskRequestsURIs], [associatedETags]]
    }

    Get a Task Request:
GET /applications/:app/taskRequests/:taskRequestID No parameters are ex- pected apart from authentication information (if needed).
Access Control. The owner of the task request or an admin.
Success. Returns error code 200 together with the JSON document of the latest
version of the task request accompanied by the ETag of the document.
Failure. Returns an error code (403, 404) together with an optional error message.
Get the Head of a Task Request:
HEAD /applications/:app/taskRequests/:taskRequestID Similar to
GET /applications/:app/taskRequests/:taskRequestIDexceptthatthebodyreturned is empty. It just returns the ETag of the latest version of the task request to indicate if there has been a change to the document and thus we need to retrieve its latest version. The access control policy is similar as above; the owner of the task request or an admin can perform the operation.
Get a Specific Version of a Task Request:
GET /applications/:app/taskRequests/:taskRequestID/v/:version No param- eters are expected apart from authentication purposes (if needed).
Access Control. The owner of the task request or an admin.
Success. Returns error code 200 together with the specific version of the task request. Failure. Returns an error code together with an optional error message.
Delete a Task Request: DELETE /applications/:app/taskRequests/:taskRequestID
This is the main URI for deleting task requests. No parameters are expected apart from authentication information (in the header). A platform job is prepared and is posted to the deletion manager.
Access Control. The owner of the task request or an admin.

B.1.2 Tasks
Tasks are generated through composition (more on that below). The most basic operations related to them are listed in the following table:
 
verb
URI
GET
/applications/:app/tasks/:taskID
HEAD
/applications/:app/tasks/:taskID
GET
/applications/:app/tasks/:taskID/v/:version
PUT
/applications/:app/tasks/:taskID
Get a Specific Task: GET /applications/:app/tasks/:taskID Similar to GET /applications/:app/taskRequests/:taskRequestID but referring to tasks. No param- eters are expected apart from authentication information (if needed).
Access Control. The participants of the task or an admin.
Success. Returns error code 200 together with the JSON document of the latest
version of the task accompanied by the ETag of the document.
Failure. Returns an error code (403, 404) together with an optional error message.
Get the Head of a Task: HEAD /applications/:app/tasks/:taskID Similar as above but the body of the response is empty. Essentially this is an easy way for the clients to figure out if the resource has changed. Same access control policy as above.
Get a Specific Version of a Task:
GET /applications/:app/tasks/:taskID/v/:version No parameters are expected apart from authentication information (if needed).
Access Control. The participants of the task or an admin.
⃝c SmartSociety Consortium 2013-2017 31 of 57
⃝c SmartSociety Consortium 2013-2017 Deliverable D2.3 Success. Returns error code 200 together with the specific version of the task.
Failure. Returns an error code together with an optional error message.
Negotiate on a Task: PUT /applications/:app/tasks/:taskID The main call for negotiation which will trigger an additional platform call to the negotiation manager. Expects the new version of the document of the task taskID. A platform job for negotiation is prepared and is posted to the negotiation manager.
Access Control. The participants of the task or an admin.
Success. Returns error code 200 together with the new version of the task as is
dictated by the negotiation manager.
Failure. Returns an error code together with an optional error message.
B.1.3 Task Records
Task records are generated by the orchestrator once execution can start on a specific task. The most basic operations are listed in the following table:
 
verb
URI
GET
/applications/:app/taskRecords/:taskRecordID
HEAD
/applications/:app/taskRecords/:taskRecordID
GET
/applications/:app/taskRecords/:taskRecordID/v/:version
PUT
/applications/:app/taskRecords/:taskRecordID

Get a Specific Task Record:
GET /applications/:app/taskRecords/:taskRecordID No parameters are expected apart from authentication information (if needed).
Access Control. The participants of the task or an admin.
Success. Returns error code 200 together with the json document of the latest version
of the task record accompanied by the ETag of the document.

Get the Head of a Task Record:
HEAD /applications/:app/taskRecords/:taskRecordID No parameters are ex- pected apart from authentication information (if needed).
The body of the response is empty. This is another convenience function which allows an easy way for the clients to figure out if the resource has changed. Same access control policy and error codes as above.
Get a Specific Version of a Task Record:
GET /applications/:app/taskRecords/:taskRecordID/v/:version No parameters are expected apart from authentication purposes (if needed).
Access Control. The participants of the task or an admin.
Success. Returns error code 200 together with the specific version of the task. Failure. Returns an error code together with an optional error message.
Provide Execution Feedback:
PUT /applications/:app/taskRecords/:taskRecordID The main call for execu- tion which will trigger an additional platform call to the execution manager. Expects the new version of the task record document taskRecordID. A platform job for execution is prepared and is posted to the execution manager.
Access Control. The participants of the task or an admin.
Success. Returns error code 200 together with the new version of the task as dictated
by the execution manager.
Failure. Returns an error code together with an optional error message.

B.2 Composition Manager
The composition manager provides the following functionality:
verb
URI
POST
/applications/:app/compositions
GET
/applications/:app/compositions/:compositionID

Perform Composition: POST /applications/:app/compositions Expects the plat- form job with the description, for which the main ingredient is the new task request that has arrived on the platform.
Access Control. The orchestrator for the application app can make such a call.
Returns. The call always succeeds and generates a resource describing the outcome of composition. Upon completion it returns an error code 201 and the link to the document with the results of composition. Part of the description of the document with the results of the composition is the error code and message that is returned through the call POST /applications/:app/taskRequests to the client.
Get Composition Results:
GET /applications/:app/compositions/:compositionID No parameters are ex- pected.
Access Control. The orchestrator for the application app or an admin can make such a call.
Success. Returns error code 200, the JSON document with the description of the results of the composition together with the associated ETag for the document.
Failure. Returns an error code (e.g. 404 not found) together with an optional error message.
Comment. Normally such a call is expected to happen only once from the applica- tion orchestrator once the latter has received the 201 error code that the composition that was requested has been performed.

B.3 Negotiation Manager
The negotiation manager provides the following functionality.
Perform Negotiation: POST /applications/:app/negotiations Expects the plat- form job with the description, for which the main ingredient is the task on which negoti- ation is being performed.
Access Control. The orchestrator for the application app can make such a call.
Returns. The call always succeeds and generates a resource describing the outcome of negotiation. Upon completion it returns an error code 201 and the link to the document with the results of the negotiation. Part of the description of the document with the results of the negotiation is the error code and message that is returned through the call PUT /applications/:app/tasks/:taskID to the client.
Get Negotiation Results:
GET /applications/:app/negotiations/:negotiationID No parameters are expected.
Access Control. The orchestrator for the application app or an admin can make such a call.
Success. Returns error code 200, the JSON document with the description of the results of the negotiation together with the associated ETag for the document.
Failure. Returns an error code (e.g. 404 not found) together with an optional error message.
Comment. Normally such a call is expected to happen only once from the applica- tion orchestrator once the latter has received the 201 error code that the negotiation that was requested has been performed.
⃝c SmartSociety Consortium 2013-2017 35 of 57
 
verb
URI
POST
/applications/:app/negotiations
GET
/applications/:app/negotiations/:negotiationID



\subsubsection{Repository}
The OM code is available (Apache v.2) at: \url{https://gitlab.com/smartsociety/orchestration}